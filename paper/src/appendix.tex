\section{Dataset Descriptions} \label{app:info}
This section will discuss the graph
datasets used, and define the homophily ratio classifier.

Of the $ 8 $ datasets used to validate the
proposed sparse SA \textsc{GCA-SA}-based models,
$ 3 $ are citation networks (Cora, Citeseer, Pubmed), 
$ 2 $ are Wikipedia
networks (Chameleon, Squirrel), 
and the last $ 3 $ are 
WebKB networks (Cornell, Texax, Wisconsin) (\cite{jiang2024self}).

The nodes in the citation graphs correspond to papers, each
containing an academic topic as a label. Edges are citations between
papers with each node. These datasets have a significant amount of 
does due to the amount of publications, and with certain paper
having hundreds of thousands of citations of them, the number 
of edges are seamingly large as well.

Nodes in the Wikipedia graphs are Wikipedia pages,
and edges refer to reciprocal links between pages. 
The interconnectedness of these graphs are extremely 
large because of the links between each type of squirrel
and chameleon.  

Finally, the WebKB datasets were collected
from each universitys' respective computer science
department. Nodes represent web pages and edges
are hyperlinks between web pages. These have the lowest
number of nodes as the webpages are relatively small
compared to Pubmed or Wikipedia. Here is a summary of all
the datasets from (\cite{jiang2024self}):


\begin{table}
  \caption{Summary of the datasets utilized in our experiment.}
  \label{tab:sample}
  \centering
  \begin{tabular}{lllllllll}
    \toprule
    \textbf{Datasets} &
    Cora & Citese. & Pubmed & Chamele. & 
    Squirr. & Cornell. & Texas & Wiscon. \\
    \midrule
    \textsc{Hom.ratio \textit{h}} &
    0.81 & 0.74 & 0.8 & 0.23 &
    0.22 & 0.3 & 0.11 & 0.21 \\
    \textsc{\# Nodes} &
    2708 & 3327 & 19717 & 2277 &
    5201 & 183 & 183 & 251 \\
    \textsc{\# Edges} &
    5429 & 4732 & 44338 & 31421 &
    198493 & 295 & 309 & 499 \\
    \bottomrule
  \end{tabular}
\end{table}

Based on the definition of the homophily ratio, citation networks
have the highest degree of homophily while 
Wikipedia and WebKB networks display low homophily. 
In the experimentation, we wish to preserve the predictive
accuracy of the low homophily networks while improving
the runtime through sparse SA mechanisms. 

