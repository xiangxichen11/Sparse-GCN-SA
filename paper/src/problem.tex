Notation will mostly follow \citet{jiang2024self}
and should be rather unsurprising.
Let $ G = (V, E) $ be an undirected graph,
where $ V $ and $ E $ are the sets of nodes
and edges respectively.
We define $ n = |V| $ to be the number of nodes
in the graph.
Let $ \mathbf{A} \in \mathbb{R}^{n \times n} $
be the (symmetric) adjacency matrix of $ G $,
where $ A_{ij} 
=\mathbbm{1} \left\{ e_{ij} \in E \right\} $.
Denote 
$ \mathbf{X} = 
\begin{bmatrix}
  \mathbf{x}_1& \cdots & \mathbf{x}_n
\end{bmatrix}^\top
\in \mathbb{R}^{n\times d} $
as our feature matrix,
where $ \mathbf{x}_i \in \mathbb{R}^d$
corresponds to the features of node $ v_i \in V $.
Finally, for a set of classes 
$ [c] = \left\{ 1, \cdots, c \right\} $
and a subset of labelled nodes
$ V_{\text{lab}} \subset V $,
$ \mathbf{y}_{\text{lab}} \in [c]^{|V_{\text{lab}}|} $
are the corresponding labels.

Our inference task is to predict the labels of the unlabeled nodes
given positional and semantic information
($\mathbf{A}$ and $\mathbf{X}$ respectively).
More formally,
find the posterior distribution
$ 
    \mathbf{y}
    \mid \mathbf{X}, \mathbf{A}, \mathbf{y}_{\text{lab}}
$
for completed label vector $ \mathbf{y} \subset [c]^n $.

Another important definition to consider is the \emph{homophily ratio}
of a graph:
\begin{definition}[Homophily Ratio]
  For graph $ G = (V, E) $,
  the homophily ratio $ h $ is 
  \begin{flalign*}
    h = \frac{|\{(v_i,v_j) \in E\colon y_i = y_j\}|}{|E|}.
  \end{flalign*}
\end{definition}

As $h \to 1 $, 
$ G $ becomes completely homophilous (high-homophily)
and as $h \to 0 $, 
$ G $ becomes completely heterophilous (low-homophily) 
\citep{zhu2020beyond}.
For our experiments, we note the homophily ratio
corresponding to each graph.
